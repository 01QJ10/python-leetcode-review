% \lesson{1}{10th May 2025}{Python Fundamentals Review}
% ----------------------------------------------------------
% Python review notes – stripped of algebraic‑topology material
% ----------------------------------------------------------
% Preamble commands such as \usepackage{amsmath}, \usepackage{listings}, etc.
% are assumed to be in the master template.

\chapter{Python Fundamentals}

\section{Scope of this Appendix}
These condensed notes cover core Python concepts and ideas of solving some LeetCode problems when 
I was preparing for interviews.  The notes assume that you already have some degree of familiarity
with Python and these are just meant to be a recap. They are not exhaustive and do not cover all Python features or
LeetCode problems (of course not). Please feel free to let me know if you think certain parts need
to be revised, or any concepts/ problems should be added!

\section{Lists vs\ Tuples}
\begin{description}
  \item[Lists] Mutable, variable‑length: \verb|[1, 2, 3]|.  Supports in‑place modification (\verb|append|, \verb|__setitem__|, etc.).  Unhashable by design.
  \item[Tuples] Immutable, fixed‑size: \verb|(1, 2, 3)|.  Once allocated, element pointers never change.  Therefore tuples can safely cache their \verb|__hash__| after the first call:
\end{description}
\begin{python}
>>> t = (1, 2, 3)
>>> hash(t)         # hash computed & cached
>>> hash(t)         # O(1) field read, no recomputation
\end{python}

\subsection*{Why Immutability Helps}
Because the tuple’s contents cannot change, the cached hash stays valid forever, allowing tuples (and frozensets) to serve as dictionary keys or set members without risk of table corruption.

\subsection*{Memory Allocation: \texttt{malloc}}
Low‑level memory for every Python object ultimately comes from the C function \verb|malloc| 
(“memory allocate”).  CPython layers an object allocator on top (small‑object pools, arenas) 
but still calls \verb|malloc|/\verb|realloc| for large blocks.

\subsection*{Lazy Hash Caching}
When \verb|hash()| is invoked on an immutable container for the first time, CPython computes the 
hash and stores it in an internal slot. Subsequent calls are constant‑time field reads, which is 
crucial for performance in hash‑table heavy workloads (e.g., counting keys in large dictionaries).

\section{Dictionaries}
\verb|dict.get(key, default)| returns the stored value if \verb|key| exists; otherwise it yields \verb|default| without raising \verb|KeyError|.
\begin{python}
d = {'a': 1, 'b': 2}
print(d.get('d', 4))  # -> 4
\end{python}

\section{Object-Oriented Programming (OOP) in Python}
\subsection{Class vs\ Instance Attributes}
\begin{itemize}
  \item \textbf{Class attribute}: defined directly in the class body.  Shared by all instances unless shadowed.
  \item \textbf{Instance attribute}: stored in the object’s \verb|__dict__|; unique per instance.
  \item Methods are class attributes that are functions; the descriptor protocol turns them into \emph{bound methods} when accessed via an instance.
\end{itemize}
\begin{python}
class Animal:
    species = 'Unknown'   # class attribute
    def __init__(self, name):
        self.name = name  # instance attribute

dog = Animal('Buddy')
print(dog.species)  # -> 'Unknown'
dog.species = 'Dog'       # shadows class attribute for this instance
print(dog.species)  # -> 'Dog'
\end{python}

\subsection{Inheritance Syntax \texttt{class Child(Parent)}}
Declaring \verb|class Child(Parent):| puts \verb|Parent| into the new type’s method‑resolution order (MRO).  Attribute look‑up proceeds \textit{instance \(\to\) Child \(\to\) Parent \(\to\dots\)} until found or \verb|AttributeError| is raised.

\section{Shallow vs\ Deep Copy}
\begin{enumerate}
  \item \textbf{Shallow copy} (\verb|copy.copy|): duplicates the outer container, reusing references to inner objects.
  \item \textbf{Deep copy} (\verb|copy.deepcopy|): recursively duplicates the entire object graph.
\end{enumerate}
Use deep copies when you must mutate nested content without altering the original; otherwise prefer the cheaper shallow copy.

\bigskip
\begin{center}\hrulefill\end{center}
\noindent\textit{Revision checklist for interview prep}
\begin{itemize}
  \item Understand reference vs\ value semantics in CPython.
  \item Be able to predict side‑effects of slicing, copying, and in‑place operations.
  \item Explain how Python’s data model (dunder methods) enables operator overloading and custom containers.
  \item Know common pitfalls: mutable default arguments, late binding in closures, iterator exhaustion.
\end{itemize}

\bigskip
\noindent\textbf{Practice prompt}: Implement a function that deep‑copies a nested list \emph{without} using \verb|copy.deepcopy|.  Analyse time/space complexity and potential stack‑overflow issues.
